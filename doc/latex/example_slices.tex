\subsection{Usage of Fits\-Parallel\-Slices}\label{example_slices}
 How to use {\bf Fits\-Parallel\-Slices} {\rm (p.\,\pageref{class_fitsparallelslices})} to compute (e.g.) a clipped median of a set of images, to build, e.g. a superflat:

\footnotesize\begin{verbatim}  ...
  FitsParallelSlices slices(sliceSize,overlap);
  
  // // insert the files ...
  for (int i=3; i<nargs; ++i)
  {
  string fileName = args[i];
  slices.AddFile(fileName);
  }
  
//  // process 
Pixel *pixelvalues = new Pixel[slices.NFiles()];
do
  {
    for (int j=0; j<slices.SliceSize(); j++) for (int i=0; i<Nx_Image; i++) 
      {
        for (int k=0; k<nImages; k++) pixelValues[k] = (*slices[k])(i,j);
        float mean;
        mean = FArrayMedian(pixelValues, nImages);
        int j_true = slices.ImageJ(j);
        (*Flat)(i,j_true) = mean;
      }
  }  
  while (slices.LoadNextSlice());
delete [] pixelValues;
...
}\end{verbatim}\normalsize 


