\subsection{Syntax of the \char`\"{}lightfile\char`\"{}}\label{lightfile}
 A lightfile consists of SN coordinates and images. The DATEZERO is used to set the lightcurve baseline. It includes all images for all filters.

\footnotesize\begin{verbatim}DATEZERO
# Here you put the dates min and max within which you are sure
# there is supernova light (DD MM YYYY  DD MM YYYY )
20 12 2000 20 12 2001

COORDREF
# The name of the DbImage where SN coordinates are expressed
MyCoordImage

COORD
# The relative coordinates of the supernova in the COORDREF mentioned above
1061.45 1048.98

IMAGES
# The list of all DbImages in all filters
# you want to build a lightcurve from
Image1_R
Image2_R
Image1_I
Image2_I\end{verbatim}\normalsize 


The COORDREF can be any reduced image, which does not need to be processed for lightcurve building (typically a subtraction).

All images in this file should:  \begin{CompactItemize}
\item 
 be flatfielded \item 
 optionally background subtracted (but preferred) \item 
 with a GAIN of 1 (TOADGAIN key) \item 
 with proper saturation level (SATURLEV key) \item 
 with a catalog produced typically by the make\_\-catalog application \item 
 in the {\bf Db\-Image} {\rm (p.\,\pageref{class_dbimage})} system \item 
 optionally with a WCS transfo (but preferred)\end{CompactItemize}
